\subsection*{Network Compression Detection }

This project began as a class project for a networking course at C\+S\+U\+N, to detect compression along a network transmission path. It has since become a research project of its own, examining the ability to use existing network facilities, particularly I\+C\+M\+P messages, to detect compression links in an uncooperative environment.

At its core {\ttfamily N\+C\+D} is a command-\/line tool similar to {\ttfamily traceroute} or {\ttfamily ping} that researchers and network admins can use to identify compression links along a network path. We believe this technique can be broadly applied to the detection of other middle-\/box behaviors. We are, however, focusing on network compression as a viable test case. Eventually we would like N\+C\+D to have support for many types of middle-\/boxes, such as deep packet inspection or other potentially intrusive behaviors.

'N\+C\+D' works by exploiting the different processing times that high entropy and low entropy data exhibit. A low entropy packet, i.\+e. containing all 0's, should compress much better -\/ and more quickly -\/ than a high entropy packet, containing \char`\"{}random\char`\"{} data. We expect the low entropy transmission times will differ significantly from the high entropy data transmission times, and give us a reliable way to detect compression. This work follows that of Vahab Pournaghshband in \href{http://lasr.cs.ucla.edu/vahab/resources/compression_detection.pdf}{\tt End-\/to-\/\+End Detection of Compression of Traffic Flows by Intermediaries}, who is my advisor on this project.

\subsection*{Current Status \href{https://travis-ci.org/ilovepi/NCD}{\tt !\mbox{[}Build Status\mbox{]}(https\+://travis-\/ci.\+org/ilovepi/\+N\+C\+D.\+svg?branch=testing)} }

Currently {\ttfamily N\+C\+D} is under heavy development, and isn't suitable for general use. It should be considered highly experimental, and unstable at best. We hope that in the near future it will be a far more robust tool that other researchers can use and improve. But by all means feel free to explore the source code, and experiment with the tool.

Once our research is complete, and we can reliably use {\ttfamily N\+C\+D} in uncooperative environments we plan to make {\ttfamily N\+C\+D} a more fully featured, more reliable C\+L\+I tool better suite for general use. 